\documentclass[12pt,a4paper]{article}

\usepackage[serbian]{babel}


\usepackage{listings}
\usepackage{xcolor} % for setting colors

% set the default code style
\lstset{language=python,
     xleftmargin=20pt,
      basicstyle=\ttfamily,
      morekeywords={to, downto, then},
      keywordstyle=\color{blue}\ttfamily,
      stringstyle=\color{red}\ttfamily,
      commentstyle=\color{green}\ttfamily,
      morecomment=[l][\color{magenta}]{\#},
      numbers=left
}

\title{Ishrana studenata}
\author{Tijana Jevti\' c \\ Jelena Mrdak}

\setlength\parindent{0pt}

\begin{document}
\maketitle
\begin{abstract}
Koliko su bitne informacije o ishrani dana\v snjim studentima? Da li njihove navike ste\v cene jo\v s u detinjstvu uti\v cu na to koju hranu danas vole da jedu? Koliko uti\v cu njihove kulinarske ve\v stine na na\v cin ishrane?

Na ova i jo\v s mnogobrojna pitanja, poku\v sa\' cemo da odgovorimo u ovom radu.
\end{abstract}

\tableofcontents

\section{Opis skupa podataka}
Skup podataka uklju\v cuje informacije o izboru hrane, ishrani, preferencijama i ostalim informacijama dobijenim od studenata na koled\v zu. Postoji 126 odgovora. Podaci su ravni i nisu o\v ci\v s\' ceni.

\begin{itemize}
  \item GPA\\
    numeri\v cki, prosek na fakultetu
  \item Pol\\
    kategori\v cki\\
    \textbf{1} - \v zensko\\
    \textbf{2} - mu\v sko
  \item Doru\v cak\\
    Ispitanicima je ponu\dj ena slika pahuljica i krofne i treba da ka\v zu \v sta ih asocira na doru\v cak\\
    \textbf{1} - pahuljice\\
    \textbf{2} - krofna 
  \item Procena kalorija u jednom par\v cetu piletine\\
    \textbf{1} - 265\\ 
    \textbf{2} - 430\\
    \textbf{3} - 610\\
    \textbf{3} - 720
  \item Da li je bitna koli\v cina kalorija koja se konzumira dnevno\\
    \textbf{1} - ne znam koliko kalorija treba konzumirati dnevno\\
    \textbf{2} - uop\v ste nije bitno\\
    \textbf{3} - umereno je bitno\\
    \textbf{3} - veoma je bitno
  \item Procena kalorija u scone from starbucks\\
    \textbf{1} - 107 cal\\ 
    \textbf{2} - 315 cal\\ 
    \textbf{3} - 420 cal\\ 
    \textbf{3} - 980 cal
  \item Kafa\\
    Ispitanicima su ponu\dj ene dve slike i treba da ka\v zu \v sta ih asocira na kafu.\\
   \textbf{1} - creamy frapuccino\\
   \textbf{2} - espreso
  \item Comfort food\\
    Ispitanici treba da navedu od 3 do 5 comfort food.
  \item  comfort food reasons\\
    Ispitanici treba da navedu do 3 razloga za\v sto jedu comfort food? (npr. tuga, sre\' ca, bes, itd)
  \item comfort food reasons coded\\
   \textbf{1} - stres\\
   \textbf{2} - dosada\\
   \textbf{3} - depresija\\
   \textbf{4} - glad\\
   \textbf{5} - lenjost\\
   \textbf{6} - hladno vreme \\
   \textbf{7} - sre\'  ca\\ 
   \textbf{8} - gledanje televizije\\
   \textbf{9} - ni\v sta od navedenog 
\end{itemize}






% 11) cook – how often do you cook?
% 1 - Every day 
% 2 - A couple of times a week 
% 3 - Whenever I can, but that is not very often  
% 4 - I only help a little during holidays 
% 5 - Never, I really do not know my way around a kitchen

% 12) cuisine – what type of cuisine did you eat growing up?
% 1 – American
% 2 – Mexican.Spanish
% 3 – Korean/Asian
% 4 – Indian
% 5 – American inspired international dishes
% 6 – other

% (lots of cleaning needed for this variable)

% 13) diet_current – describe your current diet
% open ended – ideal for NLP
	

% 14) diet_current_coded
% (based on words used to describe the diet)

% 1 – healthy/balanced/moderated/
% 2 – unhealthy/cheap/too much/random/
% 3 – the same thing over and over
% 4 – unclear


% 15) which picture do you associate with the word “drink”?
% 1 – orange juice
% 2 – soda 





% 16) eating_changes  - Describe your eating changes since the moment you got into college?
% Open ended 

% 17) eating_changes_coded

% 1 – worse
% 2 – better
% 3 – the same
% 4 – unclear


% 18) eating_changes_coded1
% 1 – eat faster
% 2 – bigger quantity
% 3 – worse quality 
% 4 – same food
% 5 – healthier
% 6 – unclear
% 7 – drink coffee 
% 8 – less food
% 9 – more sweets
% 10 – timing 
% 11 – more carbs or snacking
% 12 – drink more water
% 13 – more variety


% 19) eating_out - frequency of eating out in a typical week 
% 1 - Never 
% 2 - 1-2 times 
% 3 - 2-3 times 
% 4 - 3-5 times 
% 5 - every day


% 20) employment – do you work? 
% 1 - yes full time 
% 2 - yes part time 
% 3 – no
% 4  - other


% 21) ethnic_food - How likely to eat ethnic food 
% 1 - very unlikely 
% 2 - unlikely 
% 3 - neutral 
% 4 - likely 
% 5 - very likely 

% 22) exercise – how often do you exercise in a regular week?
% 1 - Everyday 
% 2 - Twice or three times per week 
% 3 - Once a week
% 4 - Sometimes 
% 5 – Never

% 23) father_education – 
% 1 - less than high school 
% 2 - high school degree 
% 3 - some college degree 
% 4 - college degree 
% 5 - graduate degree 


% 24) father_profession – what is your father profession?
% Open ended

% 25) fav_cuisine - What is your favorite cuisine?
% Open ended

% 26) fav_cuisine_coded

% 0-none
% 1 – Italian/French/greek
% 2 – Spanish/mexican
% 3 – Arabic/Turkish
% 4 – asian/chineses/thai/nepal
% 5 – American
% 6 – African 
% 7 – Jamaican
% 8 – indian


% 27) fav_food - was your favorite food cooked at home or store bought? 
% 1 - cooked at home 
% 2 - store bought 
% 3 - both bought at store and cooked at home

% 28) food_childhood – what was your favorite childhood food?
% Open ended


% 29) which of these pictures you associate with word fries? 
% 1 – Mcdonald’s fries
% 2 – home fries




% 30) fruit_day - How likely to eat fruit in a regular day 
% 1 - very unlikely 
% 2 - unlikely 
% 3 - neutral 
% 4 - likely 
% 5 - very likely 


% 31) grade_level – 
% 1 - freshman 
% 2 -Sophomore 
% 3 - Junior 
% 4 - Senior

% 32) greek_food - How likely to eat greek food when available?
% 1 - very unlikely 
% 2 - unlikely 
% 3 - neutral 
% 4 - likely 
% 5 - very likely 

% 33) healthy_feel – how likely are you to agree with the following statement: “I feel very healthy!” ?
% 1 to 10 where 1 is strongly agree and 10 is strongly disagree - scale

% 34) healthy_meal – what is a healthy meal? Describe in 2-3 sentences.
% Open ended

% 35) ideal_diet – describe your ideal diet in 2-3 sentences
% Open ended

% 36) Ideal_diet_coded

% 1 – portion control
% 2 – adding veggies/eating healthier food/adding fruit	
% 3 – balance
% 4 – less sugar
% 5 – home cooked/organic
% 6 – current diet
% 7 – more protein
% 8 – unclear

% 37) income
% 1 - less than $15,000 
% 2 - $15,001 to $30,000 
% 3 - $30,001 to $50,000 
% 4 - $50,001 to $70,000 
% 5 - $70,001 to $100,000 
% 6 - higher than $100,000

% 38) indian_food – how likely are you to eat indian food when available
% 1 - very unlikely 
% 2 - unlikely 
% 3 - neutral 
% 4 - likely 
% 5 - very likely 


% 39) Italian_food – how likely are you to eat Italian food when available?

% 1 - very unlikely 
% 2 - unlikely 
% 3 - neutral 
% 4 - likely 
% 5 - very likely 

% 40) life_rewarding – how likely are you to agree with the following statement: “I feel life is very rewarding!” ?
% 1 to 10 where 1 is strongly agree and 10 is strongly disagree - scale

% 41) marital_status
% 1 -Single 
% 2 - In a relationship 
% 3 - Cohabiting 
% 4 - Married 
% 5 - Divorced 
% 6 - Widowed


% 42) meals_dinner_friend – What would you serve to a friend for dinner?
% Open ended

% 43) mothers_education 
% 1 - less than high school 
% 2 - high school degree 
% 3 - some college degree 
% 4 - college degree 
% 5 - graduate degree

% 44) mothers_profession – what is your mother’s profession? 

% 45) nutritional_check - checking nutritional values frequency 
% 1 - never 
% 2 - on certain products only 
% 3 - very rarely 
% 4 - on most products 
% 5 - on everything

% 46) on_off_campus – living situation
% 1 - On campus 
% 2 - Rent out of campus 
% 3 - Live with my parents and commute 
% 4 - Own my own house

% 47) parents_cook - Approximately how many days a week did your parents cook? 
% 1 - Almost everyday 
% 2 - 2-3 times a week 
% 3 - 1-2 times a week 
% 4 - on holidays only 
% 5 - never

% 48) pay_meal_out - How much would you pay for meal out? 
% 1 - up to $5.00 
% 2 - $5.01 to $10.00 
% 3 - $10.01 to $20.00 
% 4 - $20.01 to $30.00 
% 5 - $30.01 to $40.00 
% 6 - more than $40.01 

% 49) Persian_food - How likely to eat persian food when available?
% 1 - very unlikely 
% 2 - unlikely 
% 3 - neutral 
% 4 - likely 
% 5 - very likely 

% 50) self_perception_weight - self perception of weight 
% 6 - i dont think myself in these terms 
% 5 - overweight 
% 4 - slightly overweight 
% 3 - just right 
% 2 - very fit 
% 1 - slim 

% 51) Which of the two pictures you associate with the word soup?
% 1 – veggie soup
% 2 – creamy soup





% 52) sports - sports – do you do any sporting activity?
 
% 1 - Yes 
% 2 - No 
% 99 – no answer

% 53) thai_food - How likely to eat thai food when available?
% 1 - very unlikely 
% 2 - unlikely 
% 3 - neutral 
% 4 - likely 
% 5 - very likely 

% 54) tortilla_calories - guessing calories in a burrito sandwhich from Chipolte?

% 1 - 580 
% 2 - 725 
% 3 - 940 
% 4 - 1165

% 55) turkey_calories - Can you guess how many calories are in the foods shown below? (Panera Bread Roasted Turkey and Avocado BLT)

% 1 - 345 
% 2 - 500 
% 3 - 690 
% 4 - 850


% 56) type_sports – what type of sports are you involved?
% Open-ended

% 57) veggies_day - How likely to eat veggies in a day? 
% 1 - very unlikely 
% 2 - unlikely 
% 3 - neutral 
% 4- likely 
% 5 - very likely

% 58) vitamins – do you take any supplements or vitamins?
% 1 – yes
% 2 – no

% 59) waffle_calories - guessing calories in waffle potato sandwhich 
% 1 - 575 
% 2 - 760 
% 3 - 900 
% 4 - 1315

% 60) weight – what is your weight in pounds? 



\section{Glavni deo}
u \cite{knjiga1}

\begin{lstlisting}[mathescape=true]
repeat
  i = Random(1, n)
  if x[i] == 1 then return i
\end{lstlisting}


\begin{thebibliography}{9}
  \bibitem{knjiga1}Knjiga1
  \bibitem{knjiga2}Knjiga2
\end{thebibliography}

\end{document}
